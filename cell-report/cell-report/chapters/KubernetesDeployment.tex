\chapter{Deployment Environment}
\label{chap:deployment}

\section{Infrastructure Tools}
The 5G platform is orchestrated using standard DevOps tools:
\begin{itemize}
    \item \textbf{Minikube}: Provides a local Kubernetes cluster (Docker driver). Resource allocation: 8 vCPUs, 16GB RAM.
    \item \textbf{Ansible}: Automates the deployment of pods, services, and configurations.
    \item \textbf{Helm}: Manages package deployment for OAI charts.
\end{itemize}

% \begin{figure}
%     \centering
%     \includegraphics[width=0.5\linewidth]{getpods.png}
%     \caption{Enter Caption}
%     \label{fig:placeholder}
% \end{figure}
\begin{figure}[H]
    \centering
    % \fbox{\parbox{0.85\textwidth}{\centering\vspace{3cm}\textbf{[PLACEHOLDER: Kubernetes Pods Screenshot]}\\\textit{Screenshot of `kubectl get pods -n blueprint` showing all OAI components running}\vspace{3cm}}}
    \includegraphics[width=1\linewidth]{getpods.png}
    \caption{Running pods in the `blueprint` namespace.}
    \label{fig:k8s_pods}
\end{figure}

\section{Deployment Strategy}
All components run in the `blueprint` namespace. The deployment sequence ensures dependency resolution:
\begin{enumerate}
    \item \textbf{Core Network}: MySQL $\rightarrow$ NRF $\rightarrow$ UDR/UDM/AUSF $\rightarrow$ AMF/SMF/UPF.
    \item \textbf{RIC}: FlexRIC service (exposing E2 SCTP ports 36421).
    \item \textbf{RAN}: gNodeB (connects to AMF & FlexRIC) $\rightarrow$ NR-UE (connects to gNB).
\end{enumerate}

\section{Configuration Highlights}
\subsection{Resource Limits}
To ensure stability on the local cluster, critical components have reserved resources:
\begin{itemize}
    \item \textbf{gNodeB}: 4 CPU cores, 4GB RAM (CPU intensive for RFSIM).
    \item \textbf{Core Functions}: 0.5 CPU, 512MB RAM each.
\end{itemize}

\subsection{Networking}
\begin{itemize}
    \item \textbf{ClusterIP Services}: Enable internal communication (e.g., `oai-amf` on port 38412 for NGAP).
    \item \textbf{Multus CNI}: (Optional) Can be used for separating control/user plane interfaces, though standard bridge networking is sufficient for basic monitoring.
\end{itemize}
