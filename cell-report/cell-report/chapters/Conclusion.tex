\chapter{Conclusion}
\label{chap:conclusion}

This project successfully demonstrated how to build and monitor a 5G network using open-source tools. We deployed the full OAI stack on Kubernetes and wrote a custom xApp that collects real-time data from the radio network.

\section{Key Achievements}
\begin{itemize}
    \item \textbf{Infrastructure}: We got the 5G Core and gNodeB running on Minikube.
    \item \textbf{Integration}: We connected FlexRIC to the gNodeB using the E2 interface.
    \item \textbf{Monitoring}: Our xApp successfully records 41 different metrics (like throughput and signal quality) 100 times per second.
\end{itemize}

\section{Future Work}
In the future, we could improve this by:
\begin{enumerate}
    \item Testing with real radio hardware (USRP) instead of simulation.
    \item Adding more UEs to generate complex traffic patterns.
    \item Using the collected data to train a Machine Learning model for network optimization.
\end{enumerate}

Overall, this project provided hands-on experience with 5G architecture and O-RAN concepts.
