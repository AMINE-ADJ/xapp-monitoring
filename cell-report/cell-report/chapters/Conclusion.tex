\chapter{Conclusion}
\label{chap:conclusion}

This project has successfully demonstrated the end-to-end deployment and monitoring of a 5G Standalone network using the OpenAirInterface (OAI) stack on a Kubernetes infrastructure. By integrating the FlexRIC controller via the O-RAN E2 interface, we established a robust platform for real-time network intelligence. A primary achievement was the development of a custom C-based xApp capable of capturing high-fidelity data from the Radio Access Network. This monitor successfully records 41 distinct metrics—including throughput, signal quality (RSRP/SINR), and buffer status—at a 100ms granularity, providing a comprehensive dataset for network analysis.

Looking forward, there are several avenues to expand upon this foundation. The current simulation-based setup could be validated against real-world conditions by deploying the stack on USRP software-defined radios. Furthermore, introducing a larger number of User Equipment (UEs) would enable the generation of complex, multi-user traffic patterns, stressing the network to its limits. Ultimately, the high-quality dataset generated by this project serves as a critical enabler for training Machine Learning models, paving the way for AI-driven network optimization and intelligent slicing. In conclusion, this work provided valuable hands-on experience with modern 5G architectures, solidifying our understanding of O-RAN concepts and their practical application in next-generation networks.
