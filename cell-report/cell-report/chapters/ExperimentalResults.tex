\chapter{Results \& Dataset}
\label{chap:results}

\section{Experiment Configuration}
We executed the data collection in the Minikube environment with the following parameters:
\begin{table}[H]
\centering
\begin{tabular}{|l|l|}
\hline
\textbf{Parameter} & \textbf{Value} \\
\hline
Duration & 10 seconds \\
\hline
Sample Rate & 100 Hz (1000 samples total) \\
\hline
Features & 41 per sample (Covering PHY, MAC, RLC, PDCP) \\
\hline
Channel & RFSIM Ideal Channel (High SNR, No Fading) \\
\hline
\end{tabular}
\caption{Dataset generation parameters.}
\end{table}

\section{Dataset Analysis}
The gathered dataset provides a 360-degree view of the RAN performance.

\subsection{Signal Quality (SNR)}
Since we run in simulation mode, the Signal-to-Noise Ratio (SNR) is stable and high ($\approx 50$ dB), confirming ideal channel conditions.

\subsection{Throughput \& Resource Usage}
We observed a direct correlation between allocated Physical Resource Blocks (PRBs) and user throughput.
\begin{itemize}
    \item \textbf{Downlink Throughput}: Mean $\approx 42$ Mbps.
    \item \textbf{Uplink Throughput}: Mean $\approx 38$ Mbps.
    \item \textbf{BLER}: 0\% (Expected for ideal RFSIM conditions).
\end{itemize}

% \begin{figure}[H]
%     \centering
%     \fbox{\parbox{0.8\textwidth}{\centering\vspace{3.5cm}\textbf{[PLACEHOLDER: Throughput Graph]}\\\textit{Time-series plot of DL/UL Throughput}\vspace{3.5cm}}}
%     \caption{Throughput measurements over 10 seconds.}
%     \label{fig:throughput}
% \end{figure}

\section{Conclusion on Data Quality}
The dataset successfully captures all requested metrics. The consistency of the data (zero packet loss, stable modulation MCS=9) validates the platform's stability. This clean dataset serves as a baseline for future experiments where impairments (e.g., noise, mobility) can be introduced.
