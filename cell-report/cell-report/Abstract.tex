\chapter*{Abstract}
\addcontentsline{toc}{chapter}{Abstract}

This report presents the design, implementation, and validation of a comprehensive 5G experimentation platform integrating OpenAirInterface (OAI) network components with FlexRIC near-Real-Time Radio Access Network Intelligent Controller (near-RT RIC). The platform is deployed on Kubernetes, providing a containerized environment for reproducible experiments.

The architecture comprises a complete OAI 5G Core Network with Service-Based Architecture (AMF, SMF, UPF, UDM, UDR, NRF, AUSF), an OAI gNodeB with integrated E2 Agent for O-RAN compliance, and FlexRIC providing the E2 interface termination and xApp runtime environment. All components operate within a single Kubernetes namespace, demonstrating a practical deployment model for 5G research infrastructure.

A custom xApp was developed in C to collect Key Performance Metrics (KPMs) from the RAN. The xApp subscribes to multiple O-RAN Service Models---MAC, RLC, PDCP, GTP, and KPM---enabling comprehensive cross-layer monitoring. High-frequency data collection at 100 samples per second captures 41 distinct metrics including signal quality indicators (PUSCH/PUCCH SNR), scheduling parameters (MCS, TBS, PRB allocation), block error rates (BLER), and layer-2 statistics (RLC/PDCP PDU counts, retransmissions).

Experimental validation demonstrates successful E2 interface operation, reliable Service Model indication delivery, and collection of metrics consistent with expected values for the RFSIM simulation environment. The collected dataset provides a foundation for advanced applications including machine learning-based network optimization, anomaly detection, and predictive maintenance.

\vspace{0.5cm}
\noindent\textbf{Keywords:} 5G, O-RAN, FlexRIC, xApp, KPM, OpenAirInterface, Kubernetes, RAN Monitoring, E2 Interface, Service Models

