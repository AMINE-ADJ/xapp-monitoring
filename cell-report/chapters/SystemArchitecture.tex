\chapter{System Architecture}
\label{chap:architecture}

This chapter presents the comprehensive architecture of the cloud-native 5G experimentation platform developed for research on O-RAN compliant near-real-time RAN Intelligent Controller (near-RT RIC) and Key Performance Measurement (KPM) based monitoring. The platform integrates OpenAirInterface (OAI) components for both the 5G Core Network and Radio Access Network, orchestrated through Kubernetes, with FlexRIC providing the O-RAN compliant RIC functionality.

\section{High-Level Architecture Overview}

The experimental platform consists of four primary subsystems that work in concert to provide a complete 5G network with intelligent RAN monitoring capabilities:

\begin{enumerate}
    \item \textbf{5G Core Network (5GC)}: Implements the 3GPP Release 15/16 Service-Based Architecture (SBA) using OpenAirInterface CN5G components.
    \item \textbf{5G Radio Access Network (RAN)}: Comprises the gNodeB (gNB) and User Equipment (UE) implemented using OpenAirInterface 5G RAN.
    \item \textbf{Near-RT RIC}: FlexRIC platform providing O-RAN compliant E2 interface and xApp hosting capabilities.
    \item \textbf{Orchestration Layer}: Kubernetes (Minikube) providing container orchestration, service discovery, and network management.
\end{enumerate}

\begin{figure}[H]
    \centering
    % \includegraphics[width=0.95\textwidth]{figures/high_level_architecture.png}
    \fbox{\parbox{0.9\textwidth}{\centering\vspace{3cm}\textbf{[PLACEHOLDER: High-Level System Architecture Diagram]}\\\textit{Insert diagram showing 5GC, RAN, FlexRIC, and Kubernetes interconnections}\vspace{3cm}}}
    \caption{High-level architecture of the cloud-native 5G experimentation platform with O-RAN compliant monitoring.}
    \label{fig:high_level_arch}
\end{figure}

\section{OpenAirInterface 5G Core Network}

The 5G Core Network is deployed following the 3GPP Service-Based Architecture, where Network Functions (NFs) communicate via HTTP/2 based service interfaces. The deployment utilizes OAI CN5G Federation version 1.5.1, comprising the following Network Functions:

\subsection{Core Network Functions}

\begin{table}[H]
\centering
\caption{5G Core Network Functions and their roles}
\label{tab:core_nfs}
\begin{tabular}{|l|l|p{7cm}|}
\hline
\textbf{NF} & \textbf{Full Name} & \textbf{Primary Function} \\
\hline
AMF & Access and Mobility Management Function & Handles UE registration, connection management, mobility, and NAS signaling \\
\hline
SMF & Session Management Function & Manages PDU sessions, UPF selection, and QoS policies \\
\hline
UPF & User Plane Function & Forwards user data packets, applies QoS, and handles N3/N6 interfaces \\
\hline
UDM & Unified Data Management & Manages subscriber data and authentication credentials \\
\hline
UDR & Unified Data Repository & Stores subscription and policy data \\
\hline
NRF & Network Repository Function & Service discovery and NF profile management \\
\hline
AUSF & Authentication Server Function & Handles UE authentication procedures \\
\hline
\end{tabular}
\end{table}

\subsection{Core Network Interfaces}

The 5G Core implements the following reference points as defined by 3GPP TS 23.501:

\begin{itemize}
    \item \textbf{N1}: Between UE and AMF for NAS signaling
    \item \textbf{N2}: Between RAN (gNB) and AMF for NGAP control plane
    \item \textbf{N3}: Between RAN (gNB) and UPF for GTP-U user plane
    \item \textbf{N4}: Between SMF and UPF for PFCP session management
    \item \textbf{N6}: Between UPF and external Data Network (DN)
\end{itemize}

\section{OpenAirInterface 5G RAN}

The Radio Access Network implementation utilizes OpenAirInterface 5G, providing a software-defined gNodeB and NR User Equipment. The RAN operates in standalone (SA) mode with the following configuration:

\subsection{gNodeB Configuration}

\begin{table}[H]
\centering
\caption{gNodeB radio configuration parameters}
\label{tab:gnb_config}
\begin{tabular}{|l|l|}
\hline
\textbf{Parameter} & \textbf{Value} \\
\hline
Frequency Band & n78 (3.5 GHz TDD) \\
\hline
Bandwidth & 40 MHz (106 PRBs) \\
\hline
Subcarrier Spacing & 30 kHz \\
\hline
Duplex Mode & Time Division Duplex (TDD) \\
\hline
MIMO Configuration & 1x1 SISO \\
\hline
Physical Layer & RF Simulator (RFSIM) \\
\hline
PLMN & 001/01 \\
\hline
Cell ID (NCI) & 0xe000 \\
\hline
\end{tabular}
\end{table}

\subsection{E2 Agent Integration}

The OAI gNB includes an integrated E2 Agent that establishes connectivity with the near-RT RIC via the E2AP protocol. The E2 Agent exposes multiple RAN Functions (RF) through Service Models:

\begin{table}[H]
\centering
\caption{Supported E2 Service Models in OAI gNB}
\label{tab:service_models}
\begin{tabular}{|l|l|l|}
\hline
\textbf{SM ID} & \textbf{Service Model} & \textbf{Description} \\
\hline
2 & ORAN-E2SM-KPM & Key Performance Measurement \\
\hline
3 & ORAN-E2SM-RC & RAN Control \\
\hline
142 & MAC\_STATS\_V0 & MAC layer statistics \\
\hline
143 & RLC\_STATS\_V0 & RLC layer statistics \\
\hline
144 & PDCP\_STATS\_V0 & PDCP layer statistics \\
\hline
145 & SLICE\_STATS\_V0 & Network slicing statistics \\
\hline
146 & TC\_STATS\_V0 & Traffic control statistics \\
\hline
148 & GTP\_STATS\_V0 & GTP tunnel statistics \\
\hline
\end{tabular}
\end{table}

\section{FlexRIC Near-RT RIC Platform}

FlexRIC serves as the O-RAN compliant near-Real-Time RAN Intelligent Controller, providing the interface between RAN elements and intelligent applications (xApps). The platform implements the E2 interface as specified by the O-RAN Alliance.

\subsection{FlexRIC Components}

The FlexRIC deployment comprises:

\begin{itemize}
    \item \textbf{Near-RT RIC Core}: Handles E2AP message processing, subscription management, and indication routing
    \item \textbf{E2 Termination}: SCTP-based endpoint for E2 connections from E2 Nodes (gNB)
    \item \textbf{xApp SDK}: C and Python libraries for xApp development
    \item \textbf{Service Model Plugins}: Dynamically loaded libraries for SM encoding/decoding
\end{itemize}

\subsection{E2 Interface}

The E2 interface follows the O-RAN E2AP specification, utilizing SCTP for reliable message delivery. Key procedures include:

\begin{itemize}
    \item \textbf{E2 Setup}: Establishes association between E2 Node and near-RT RIC
    \item \textbf{RIC Subscription}: xApps subscribe to specific RAN Functions for periodic indications
    \item \textbf{RIC Indication}: Delivers measurement data from E2 Node to xApps
    \item \textbf{RIC Control}: Enables xApps to send control commands to the RAN
\end{itemize}

\begin{figure}[H]
    \centering
    % \includegraphics[width=0.85\textwidth]{figures/e2_interface.png}
    \fbox{\parbox{0.85\textwidth}{\centering\vspace{2.5cm}\textbf{[PLACEHOLDER: E2 Interface Protocol Stack]}\\\textit{Insert diagram showing E2AP over SCTP, Service Models, and RAN Functions}\vspace{2.5cm}}}
    \caption{E2 interface architecture between OAI gNB and FlexRIC near-RT RIC.}
    \label{fig:e2_interface}
\end{figure}

\section{Component Interconnection}

Figure~\ref{fig:component_interconnect} illustrates the complete data and control flow between all system components. The architecture enables:

\begin{itemize}
    \item Real-time performance metric collection from the RAN
    \item Service-based communication within the 5G Core
    \item Container orchestration through Kubernetes
    \item Intelligent monitoring through xApps
\end{itemize}

\begin{figure}[H]
    \centering
    % \includegraphics[width=0.95\textwidth]{figures/component_interconnection.png}
    \fbox{\parbox{0.9\textwidth}{\centering\vspace{4cm}\textbf{[PLACEHOLDER: Detailed Component Interconnection Diagram]}\\\textit{Insert comprehensive diagram showing all NFs, gNB, UE, FlexRIC, and xApp connections with interface labels}\vspace{4cm}}}
    \caption{Detailed component interconnection showing control plane and user plane data flows.}
    \label{fig:component_interconnect}
\end{figure}
