\chapter{Conclusion}
\label{chap:conclusion}

This report has presented a comprehensive 5G experimentation platform integrating OpenAirInterface (OAI), FlexRIC, and a custom xApp for Key Performance Metric (KPM) data collection. The platform provides an end-to-end environment for 5G network research, monitoring, and optimization.

\section{Summary of Contributions}

\subsection{Platform Architecture}

The developed platform successfully integrates:

\begin{itemize}
    \item \textbf{OAI 5G Core Network}: A complete 3GPP-compliant 5G Core with AMF, SMF, UPF, UDM, UDR, NRF, and AUSF network functions, supporting the 5G Service-Based Architecture and providing subscriber management, session handling, and user plane connectivity.
    
    \item \textbf{OAI 5G RAN}: A software-defined gNodeB operating in RFSIM mode with integrated E2 Agent for O-RAN compatibility, supporting the n78 frequency band with 40 MHz bandwidth and 30 kHz subcarrier spacing.
    
    \item \textbf{FlexRIC Near-RT RIC}: An O-RAN compliant near-Real-Time RIC providing the E2 interface for RAN control and monitoring, supporting multiple Service Models including MAC, RLC, PDCP, GTP, and KPM.
    
    \item \textbf{Kubernetes Orchestration}: A containerized deployment architecture enabling reproducible deployments, resource management, and simplified operations across all platform components.
\end{itemize}

\subsection{xApp Development}

A custom xApp for KPM data collection was designed and implemented, featuring:

\begin{itemize}
    \item Native C implementation leveraging the FlexRIC SDK for optimal performance
    \item Multi-Service Model subscription for comprehensive metric coverage
    \item High-frequency data collection at 100 samples per second
    \item Structured CSV output for integration with analysis tools
    \item 41 distinct metrics spanning PHY, MAC, RLC, PDCP, GTP, and KPM layers
\end{itemize}

\subsection{Experimental Validation}

The platform was validated through systematic data collection, demonstrating:

\begin{itemize}
    \item Successful E2 interface operation between gNB and FlexRIC
    \item Reliable Service Model indication delivery
    \item Consistent metric collection without data loss
    \item Expected metric values and statistical properties
\end{itemize}

\section{Platform Capabilities}

The developed platform enables various research and development activities:

\begin{enumerate}
    \item \textbf{RAN Monitoring}: Real-time visibility into radio interface performance through comprehensive KPM collection.
    
    \item \textbf{Algorithm Development}: A testbed for developing and evaluating scheduling, resource allocation, and optimization algorithms.
    
    \item \textbf{Machine Learning Applications}: High-quality datasets for training ML models for anomaly detection, traffic prediction, and network optimization.
    
    \item \textbf{Protocol Analysis}: Cross-layer visibility for protocol behavior analysis and troubleshooting.
    
    \item \textbf{Education and Training}: A complete 5G environment for educational purposes and hands-on learning.
\end{enumerate}

\section{Future Work}

Several directions for future enhancement are identified:

\subsection{Platform Extensions}

\begin{itemize}
    \item \textbf{Multi-UE Support}: Extend the platform to support multiple concurrent UEs for more realistic traffic scenarios.
    
    \item \textbf{Hardware Integration}: Transition from RFSIM to real RF hardware (USRP) for over-the-air experiments.
    
    \item \textbf{Core Network Monitoring}: Develop xApps for monitoring 5G Core network functions and N2/N4 interfaces.
    
    \item \textbf{Network Slicing}: Implement network slicing capabilities with per-slice monitoring and control.
\end{itemize}

\subsection{xApp Enhancements}

\begin{itemize}
    \item \textbf{Control Loop Implementation}: Extend the xApp to implement closed-loop control using the RC (RAN Control) Service Model.
    
    \item \textbf{Real-Time Analytics}: Integrate stream processing for real-time anomaly detection and alerting.
    
    \item \textbf{ML Model Integration}: Embed trained ML models for online inference and automated optimization.
    
    \item \textbf{Multi-Cell Support}: Scale the xApp to monitor multiple gNBs simultaneously.
\end{itemize}

\subsection{Research Directions}

\begin{itemize}
    \item \textbf{QoS Optimization}: Develop algorithms for dynamic QoS management based on collected metrics.
    
    \item \textbf{Energy Efficiency}: Investigate RAN energy consumption optimization through intelligent resource management.
    
    \item \textbf{Interference Management}: Implement coordinated multi-point techniques using FlexRIC control capabilities.
\end{itemize}

\section{Concluding Remarks}

This work demonstrates the feasibility of building a complete, functional 5G experimentation platform using open-source components. The integration of OAI, FlexRIC, and Kubernetes provides a flexible and extensible foundation for 5G research. The developed KPM Data Collector xApp validates the platform's monitoring capabilities and provides valuable datasets for further analysis.

The O-RAN architecture's separation of concerns---between RAN intelligence and radio protocol processing---enables modular development of intelligent network applications. The platform presented in this report serves as a reference implementation for researchers and practitioners working on next-generation mobile network optimization and automation.

\vspace{1cm}

\begin{center}
\rule{0.5\textwidth}{0.4pt}
\end{center}
